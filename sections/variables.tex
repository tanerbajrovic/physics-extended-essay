\documentclass[../main.tex]{subfiles}

\begin{document}
\doublespacing

\subsubsection{Independent Variable}

Mass concentration, $\rho_{i}$, is a \textbf{an amount of substance} within a given \textbf{mixture}, respectively referred to as a solute and a solvent. Mathematically, it follows:
\begin{equation} \label{eq:concentration}
    \rho_{i}= \frac{m_{solute}}{V}
\end{equation}
where $m_{solute}$ is mass of the solute (salt) and $V$ is the volume of the solvent (water). Mass concentration, $\rho_i$, is the independent variable and is changed by changing the mass of the table salt, $m_{solute}$, while the volume $V$ remains constant. It follows that by increasing the value of $m_{solute}$ in the range of $\SI{1.0}{\gram} - \SI{5.0}{\gram}$, the value of $\rho_i$ increases as well. % Consequently this was achieved 

\subsubsection{Dependent Variable}

In as such, the cooling rate of the sample, $\dv{T}{t}$, represents a dependent variable as it is affected by other factors, namely salt concentration of a given sample as well as the initial temperature, $T_i$. It follows from the hypothesis that by increasing the concentration of salt, the value of $\dv{T}{t}$ will increase as well.

% Additionally, elapsed time $t$ until the water cools down to \SI{0}{\celsius} constitutes another dependent variable since 

\subsubsection{Controlled Variables}

\begin{table}[H]
\setlength\extrarowheight{5pt}
    \centering
    \begin{tabular}[H]{|M{5cm}|M{5cm}|M{5cm}|}
    \hline
    \textbf{Controlled Variables}   &   \textbf{Description}   & \textbf{Value}\\[5pt] \hline % End of Header
    \textit{Ambient Temperature} \newline $(T_{amb})$ & Affects the cooling time of the samples inside freezer. It was controlled by allowing the temperature to drop to $\SI{-20}{\celsius}$ after every finished trial. & $(-20 \pm 0.5)\si{\celsius}$ \\[5pt] \hline
    \textit{Initial Temperature \break of Water} \break $(T_{i})$ & Initial temperatures of $T_i = \SI{28}{\celsius}$ and $T_i = \SI{50}{\celsius}$ were used as starting values for the measurements. Sample was allowed to cool down until the given temperature and the data was recorded & $T_i = \SI{28}{\celsius}$ and $T_i = \SI{50}{\celsius}$ \\[10pt]\hline
    \textit{Volume of the Water} \newline $(V)$ & Volume of water directly affects the time taken to cool down a sample. All samples were $\SI{40}{\milli\liter}$ and measured using a beaker prior and after boiling. & $(40 \pm 0.5) \si{\milli\liter}$ \\[5pt]\hline
    \textit{Placement in the Freezer} & The position of the sample changes the cooling efficiency of the freezer. As such, all samples were placed \textit{approximately} in the center of the freezer. & - \\[4pt]\hline
    \textit{Placement of Thermistor inside  Water} & Thermistor was fully submerged \emph{approximately} in the middle of the the sample, \SI{1}{\centi\meter} from the surface.& - \\[5pt]\hline
    \textit{Freezer Floor Insulation} & Insulation affects the rate of transfer of heat. Paper towels were placed between the cups and the freezer floor to minimize the frost and thermally insulate the sample .& - \\[5pt]\hline
    \end{tabular}
    \caption{Controlled Variables}
    \label{tab:controlledVariables}
\end{table}


\end{document}