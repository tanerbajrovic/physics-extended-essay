\documentclass[../main.tex]{subfiles}

\begin{document}
\cleardoublepage
\phantomsection
\section*{Appendices} \label{sec:appendices}
\addcontentsline{toc}{section}{\protect Appendices}


% First Appendix
\addcontentsline{toc}{subsection}{\protect Appendix A. Apparatus Construction}
\subsection*{Apparatus Construction} \label{subsec:appendixOne}

\begin{figure}[H]
\centering
\begin{minipage}{.49\textwidth}
  \centering
  \captionsetup{justification=centering}
  \begin{circuitikz}[scale=1, european] 
        % \draw (0,0) to[battery=5V, l_=Arduino] (0,4);
        % \draw (0,4) -- (4,4)
        %     to[R, l=$10^3\ \si{\ohm}$] (4,2); % Resistor
        % \draw (4,1.7) to[short, *-o] (6,1.7) node[anchor=west] {A0};
        % \draw(4,2) to[R, label={$R_T$}] (4,0); % Thermistor
        % \draw (4,0.3) to[short, *-o] (6,0.3) node[anchor=west] {GND};
        % \draw (4,0) -- (0,0);
        
        % \draw (0,0) to[battery=5V, l_=Arduino] (0,4);
        % \draw (0,4) -- (4,4)
        \draw (4,4) to[short, *-o] (0,4) node[anchor=east] {5V};
        \draw (4,4) to[R, l=$10^3\ \si{\ohm}$] (4,2); % Resistor
        \draw (4,2) to[short, *-o] (6,2) node[anchor=west] {A0};
        \draw(4,2) to[R, label={$R_T$}] (4,0); % Thermistor
        \draw (4,0) to[short, *-o] (0,0) node[anchor=east] {GND};
    \end{circuitikz}
    \caption{Circuit of resistor and thermistor $R_T$ connected to Arduino}
    \label{fig:circuitDrawingAppendix}
\end{minipage}%
\begin{minipage}{.49\textwidth}
  \centering
  \captionsetup{justification=centering}
  \begin{tabular}{cc}
        \toprule
        Temperature $\left(\si{\celsius}\right)$ & Resistance $\left(\si{\ohm}\right)$ \\
        (\SI{\pm 0.1}{\celsius}) & (\SI{\pm 1}{\ohm}) \\
        \midrule
        $22.6$ & $1330$\\
        $11.3$ & $2140$\\
        $9.5$  & $2320$\\
        $-13.7$& $7070$\\
        \bottomrule
    \end{tabular}
  \captionof{table}{Values of $R_T$ at \\specific temperatures}
  \label{tab:resistorCharacteristicsAppendix}
\end{minipage}
\end{figure}

From Fig. \ref{fig:circuitDrawingAppendix}, whole circuit has a resistance $R_{tot} = R + R_T$, where $R$ is resistor with nominal resistance $\SI{1}{\kilo\ohm}$ (experimentally $\SI{989}{\ohm}$) and $R_T$ is the thermistor. Circuit has the potential difference $V_{arduino}$, thus:

\begin{gather}
    V_{arduino} = I \cdot \frac{1}{R + R_T} \implies I = \frac{V_{arduino}}{R + R_T} \\
\intertext{Expressing for $R_T$ and using the expression above}
V_T = I \cdot R_T \\
V_T = \left(\frac{V_{arduino}}{R + R_T}\right) \cdot R_T \implies V_T= V_{arduino} \cdot \frac{R_T}{R + R_T}.
\end{gather} \label{eq:circuitFinalAppendix}

Using the guide provided by \textcite{thermistorCalibration}, the Steinhart-Hart $\beta$ Parameter equation was used to approximate the resistance-temperature relationship for the said thermistor:

\begin{gather}
    R(T) = R_0 \cdot e^{B \left(1/T - 1/T_0\right)} \\
\intertext{where $R_0$ is the initial resistance and $T_0$ is the initial temperature from Table.~\ref{tab:resistorCharacteristicsAppendix}, $R$ is theoretical resistance and $T$ is the corresponding temperature for that resistance.}
    \frac{R}{R_0} = e^{B \left(1/T - 1/T_0\right)} \nonumber \\
    \ln{\frac{R}{R_0}} = \frac{B}{T} - \frac{B}{T_0} \nonumber \\
    \ln{\frac{R}{R_0}} + \frac{B}{T_0} = \frac{B}{T} \nonumber \\
    T = \frac{B}{\ln{\left(\frac{R}{R_0}\right)} + \frac{B}{T_0}} \nonumber \\
    T = \frac{T_0 \cdot B}{T_0 \cdot \ln{\left(\frac{R}{R_0}\right)} + B} \nonumber
\intertext{or}
    T = \frac{B}{\ln{\left(\frac{R}{R_0}\right)} + \ln{\left(e^{B/T_0}\right)}} \nonumber\\
    T = \frac{B}{\ln{\left(\frac{R}{R_0} \cdot e^{B/T_0}\right)}}. \nonumber \\
\intertext{Finally, rearranging and lastly subtracting 273.15 to convert from Kelvins into Celsius:}
    T = \frac{B}{\ln{\left(\frac{R}{R_0 \cdot e^{-B/T_0}}\right)}} \implies \boxed{T = \frac{B}{\ln{\left(\frac{R}{R_0 \cdot e^{-B/T_0}}\right)}} - 273.15.}
\end{gather} \label{eq:temperatureFinal}

Coefficient $B$ was obtained by using website \textcite{betaCalculation} and found to be $B = 3540.94$ for values shown in Table.~\ref{tab:resistorCharacteristics}. \par 

Using these formulas in \eqref{eq:circuitFinal} and \eqref{eq:temperatureFinal} Python code was written (see \textbf{Appendix: Python Code}). It firstly uses Arduino apparatus to find $R_T$, thermistor resistance, and then converts $R_T$ into temperature by using the formulas outlined above. \par 

% Second Appendix
\newpage \thispagestyle{plain}
\subsubsection*{Arduino Code} \label{subsubsec:arduinoCode} \addcontentsline{toc}{subsubsection}{\protect Arduino Code}
\lstset{breaklines=true}
\lstinputlisting[style=myArduino]{code/code.ino}

% Python Code
\newpage \thispagestyle{plain}
\subsubsection*{Python Code: Data Collection from Arduino} \label{subsubsec:pythonCode}
\addcontentsline{toc}{subsubsection}{\protect Python Code}
\lstset{breaklines=true}
\lstinputlisting[language=python]{code/pythonCode.py}
\end{document}