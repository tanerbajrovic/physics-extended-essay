\documentclass[../main.tex]{subfiles}

\begin{document}
\subsection{Evaluation}
\doublespacing
% Interpret Results
Results of the study imply that Mpemba effect was not observed for any values of mass concentration used which is in support of findings by \textcite{burridge_questioning_2016} and \textcite{burridge_observing_2020}. However, since the effect was very close to being observed for some samples, it stands to reason that possible limitations might have contributed 

% Possible limitations
Considering the design of the experiment, it may be noted that the possible limitations could arise due to a couple of factors.
Firstly, the experiment used tap water for all of its measurements. Due to impurities, tap water might have had a different freezing temperature and might have behaved differently as a result of that. Using deionised water instead could give more concrete results, as it would allow for studying the effect of NaCl (table salt) without the presence of other elements.
Secondly, using multiple instruments to measure the same sample would ensure that error does not occur due to miscalibration. Additionally, the placement of the thermistor inside the sample might influence the temperature since different parts of the liquid have different temperatures. For the sake of repeatability, the thermistor was placed approximately in the middle of the sample $\SI{1}{\centi\meter}$. Lastly, by using different freezing points (where the first ice particles form, \SI{0}{\celsius} or through definition involving latent heat) different results may be obtained. In as such, using a reference point, as in \textcite{ibekwe_investigating_2016}, might lead to the Mpemba effect being observed. This was, however, avoided to stay more in line with the original research paper.  \par

In addition to the previously noted, future studies could explore the effects of changing volumes, having a greater range of initial temperatures and using different solutes like Mg. For instance, changing a volume from $\SI{40}{\milli\liter}$ to $\SI{80}{\milli\liter}$ would impact the cooling rate, and thus might give rise to different behaviors of temperature curves. Similarly, expanding the range of initial temperatures might help identify instances in which the effect happens. Finally, exploring the approach using other solutes might prove beneficial since it considers how other elements might impact the nature of water.  \par 
\end{document}